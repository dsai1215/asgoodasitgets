\documentclass[twocolumn]{IEEEtran}
% \documentclass{colt2013} % Include author names

\usepackage{macros}
\pdfminorversion=4

\tikzstyle{block} = [draw, fill=white, rectangle,
    minimum height=3em, minimum width=6em]
\tikzstyle{sum} = [draw, fill=white, circle, node distance=1cm]
\tikzstyle{input} = [coordinate]
\tikzstyle{output} = [coordinate]
\tikzstyle{pinstyle} = [pin edge={to-,thin,black}]
\tikzset{roads/.style={line width=0.2cm}}

\usepackage{times}

% The following packages will be automatically loaded:
% amsmath, amssymb, natbib, graphicx, url, algorithm2e
\begin{document}

\title{\LARGE \bf
Improved Hessian estimation for adaptive \\random directions stochastic approximation
}



\author{D. Sai Koti Reddy$^\dagger$, Prashanth L.A.$^\sharp$, Shalabh Bhatnagar$^\ddag$
\thanks{
$^\dagger$ Department of Computer Science and Automation,
Indian Institute of Science, Bangalore,
E-Mail: danda.reddy@csa.iisc.ernet.in}
\thanks{
$^\sharp$ Institute for Systems Research, University of Maryland, College Park, Maryland,
E-Mail: prashla@isr.umd.edu.
}
\thanks{
$^\ddag$ Department of Computer Science and Automation,
Indian Institute of Science, Bangalore,
E-Mail: shalabh@csa.iisc.ernet.in
}
}

\maketitle

\begin{abstract}
\end{abstract}

\begin{IEEEkeywords}
Stochastic optimization, stochastic approximation, random directions stochastic approximation (RDSA), simultaneous perturbation stochastic approximation (SPSA).
\end{IEEEkeywords}

\section{INTRODUCTION}
\label{sec:introduction}
We consider the following \textit{simulation optimization} problem:
\begin{align}
\mbox{Find } x^* = \arg\min_{x \in \R^N} f(x). \label{eq:pb}
\end{align}
The aim is to find an algorithm that solves \eqref{eq:pb} given only noise-corrupted measurements of the objective $f$, as illustrated in Figure \ref{fig:so}.
A natural solution approach is to devise an algorithm that incrementally updates the parameter, say $x_n$, in the descent direction using the gradient and/or Hessian of the objective $f$. However, gradient/Hessian of $f$ are unavailable directly and need to be estimated from samples of $f$. 
To overcome this, we adopt the simultaneous perturbation (SP) approach - a popular and efficient idea especially in high dimensional problems - see \cite{bhatnagar-book} for a comprehensive treatment of this subject matter. 


Simultaneous perturbation stochastic approximation (SPSA) is a popular SP method. The first-order SPSA algorithm, henceforth referred to as 1SPSA, was proposed in \cite{spall}.  A closely related algorithm is random directions stochastic approximation (RDSA) \cite[pp.~58-60]{kushcla}. The gradient estimate in RDSA differs from that in SPSA, both in the construction as well as in the choice of random perturbations.  In \cite{kushcla}, the random perturbations for 1RDSA were generated by picking uniformly on the surface of a sphere and the resulting 1RDSA scheme was found to be  inferior to 1SPSA from an asymptotic convergence rate viewpoint - see \cite{chin1997comparative}.  Recent work in \cite{prashanth2015rdsa} attempts to bridge the gap between 1RDSA and 1SPSA by incorporating random perturbations based on an asymmetric Bernoulli distribution. However,  1SPSA was found to be still marginally better than 1RDSA. 
On the other hand, results in \cite{prashanth2015rdsa} show that a second order RDSA approach (referred to as 2RDSA hereafter) can considerably outperform the corresponding higher order SPSA algorithm \cite{spall_adaptive} (referred to as 2SPSA hereafter).  

Our work in this paper is centered on improving the 2RDSA scheme of \cite{prashanth2015rdsa} by \\
\begin{inparaenum}[\bfseries (i)]
\item reducing the error in the Hessian estimate through a feedback term; and\\
\item optimizing the step-sizes used in the Hessian estimation recursion, again with the objective of improving the quality of the Hessian estimate.
\end{inparaenum}

\tikzstyle{block} = [draw, fill=white, rectangle,
   minimum height=3em, minimum width=6em]
\tikzstyle{sum} = [draw, fill=white, circle, node distance=1cm]
\tikzstyle{input} = [coordinate]
\tikzstyle{output} = [coordinate]
\tikzstyle{pinstyle} = [pin edge={to-,thin,black}]

 \begin{figure}[t]
    \centering
\scalebox{0.8}{\begin{tikzpicture}[auto, node distance=2cm,>=latex']
% We start by placing the blocks
\node (theta) {$\boldsymbol{x_n}$};
\node [block, fill=blue!20,right=0.6cm of theta,align=center] (sample) {\makecell{\textbf{Measurement}\\\textbf{ Oracle}}}; 
\node [right=0.6cm of sample] (end) {$\boldsymbol{\mathbf{f(x_n) + \xi_n}}$};
\node [ above right= 0.6cm of end] (bias) {\textbf{Zero mean}};
\draw [->] (theta) --  (sample);
\draw [->] (sample) -- (end);
\path [darkgreen,->] (bias) edge [bend left] (end);
\end{tikzpicture}}
\caption{Simulation optimization}
\label{fig:so}
\end{figure}

Items (i) and (ii) are inspired by the corresponding improvements to the Hessian estimation recursion in 2SPSA - see \cite{spall-jacobian}. While item (ii) is a relatively straightforward migration to the 2RDSA setting, item (i) is a non-trivial contribution, primarily because the Hessian estimate in 2RDSA is entirely different from that in 2SPSA and the feedback term that we incorporate in 2RDSA to improve the Hessian estimate neither correlates with that in 2SPSA nor follows from the analysis in \cite{spall-jacobian}. 
The advantage with 2RDSA-IH is that it requires only 75\% of the simulation cost per-iteration for 2SPSA-IH. 

We establish that the proposed improvements to Hessian estimation in 2RDSA are such  that the resulting 2RDSA-IH algorithm is provably convergent to the true Hessian. 
%Moreover, we show, for the special case of a quadratic objective, that	2RDSA-IH results in a convergence rate that is on par with the corresponding rate for 2SPSA with Hessian estimation improvements (2SPSA-IH) \cite{spall-jacobian}. 
Further, we show empirically that 2RDSA-IH that we propose here outperforms both 2SPSA-IH of \cite{spall-jacobian} and regular 2RDSA of \cite{prashanth2015rdsa}. Our contribution is important because 2RDSA-IH, like 2RDSA, has lower simulation cost per iteration than 2SPSA and unlike 2RDSA, has an improved Hessian estimation scheme.

%The rest of the paper is organized as follows: In Section~\ref{sec:2rdsa-ih}, we
%describe the improved Hessian estimation scheme, which is incorporated into the  second-order RDSA algorithm from \cite{prashanth2015rdsa}. In Section \ref{sec:2rdsa-results}, we present the theoretical results for the second-order RDSA algorithm with our improved Hessian estimation scheme.
%In Section~\ref{sec:expts}, we present the results from numerical experiments and finally, in  Section~\ref{sec:conclusions}, provide the concluding remarks.


\section{SECOND-ORDER RDSA WITH IMPROVED HESSIAN ESTIMATION (2RDSA-IH)}
\label{sec:2rdsa-ih}
The second-order RDSA with improved Hessian estimate performs an update iteration as follows:
\begin{align}
\label{eq:e2rdsa}
x_{n+1} = & \; x_n - a_n \Upsilon(\overline H_n)^{-1}\widehat\nabla f(x_n), \\
\overline H_n = & \; (1-b_{n})  \overline H_{n-1} + b_{n} ( \widehat H_n - \widehat \Psi_n),\label{eq:2rdsa-H}
\end{align}
where $\widehat\nabla f(x_n)$ is the estimate of $\nabla f(x_n)$, 
 %and this corresponds to  the uniform  and   asymmetric Bernoulli variant gradient estimates proposed in \cite{prashanth2015rdsa}.
$\overline H_n$ is an estimate of the true Hessian ${\nabla}^2 f(\cdot)$, $\Upsilon(\cdot)$ projects any matrix onto the set of positive definite matrices and $\{a_n, n\ge 0\}$ is a step-size sequence that satisfies standard stochastic approximation conditions. There are standard procedures such as Cholesky factorization, see \cite{bert22}, for projecting a given square matrix to set of positive definite matrices. Moreover, in the vicinity of a local minimum, one expects the Hessian to be positive definite. In such a case, $\Upsilon$ will represent the identity operator.

The recursion \eqref{eq:e2rdsa} is identical to that in 2RDSA, while the Hessian estimation recursion \eqref{eq:2rdsa-H} differs as follows:\\
\begin{inparaenum}[\bfseries (i)]
\item  $\widehat \Psi_n$ is a zero-mean feedback term that reduces the error in Hessian estimate; and\\
\item $b_n$ is a general step-size that we optimize to improve the Hessian estimate.
\end{inparaenum}

On the other hand, $\widehat H_n$ is identical to that in 2RDSA, i.e., it estimates the true Hessian in each iteration using $3$ function evaluations. %Further, $\widehat H_0 = I$. 
For the sake of completeness, we provide below the construction for $\widehat\nabla f(x_n)$ and $\widehat H_n$ using asymmetric Bernoulli perturbations, before we present the feedback term that reduces the error in $\widehat H_n$.

%%%%%%%%%%%%%%%%%%%%%%%%%%%%%%%%%%%%%%%%%%%%%%%%%%%%%%%%%%%%%%%%%%%%%%%%%%%%%%%%%%%%%%
%%%%%%%%%%%%%%%%%%%%%%%%%%%%%%%%%%%%%%%%%%%%%%%%%%%%%%%%%%%%%%%%%%%%%%%%%%%%%%%%%%%%%%
\algblock{PEval}{EndPEval}
\algnewcommand\algorithmicPEval{\textbf{\em Function evaluation 1}}
 \algnewcommand\algorithmicendPEval{}
\algrenewtext{PEval}[1]{\algorithmicPEval\ #1}
\algrenewtext{EndPEval}{\algorithmicendPEval}

\algblock{PEvalPrime}{EndPEvalPrime}
\algnewcommand\algorithmicPEvalPrime{\textbf{\em Function evaluation 2}}
 \algnewcommand\algorithmicendPEvalPrime{}
\algrenewtext{PEvalPrime}[1]{\algorithmicPEvalPrime\ #1}
\algrenewtext{EndPEvalPrime}{\algorithmicendPEvalPrime}

\algblock{PImp}{EndPImp}
\algnewcommand\algorithmicPImp{\textbf{\em Gradient descent}}
 \algnewcommand\algorithmicendPImp{}
\algrenewtext{PImp}[1]{\algorithmicPImp\ #1}
\algrenewtext{EndPImp}{\algorithmicendPImp}

\algtext*{EndPEval}
\algtext*{EndPEvalPrime}
\algtext*{EndPImp}
%%%%%%%%%%%%%%%% alg-custom-block %%%%%%%%%%%%
\algblock{PEvalPrimeDouble}{EndPEvalPrimeDouble}
\algnewcommand\algorithmicPEvalPrimeDouble{\textbf{\em Function evaluation 3}}
 \algnewcommand\algorithmicendPEvalPrimeDouble{}
\algrenewtext{PEvalPrimeDouble}[1]{\algorithmicPEvalPrimeDouble\ #1}
\algrenewtext{EndPEvalPrimeDouble}{\algorithmicendPEvalPrimeDouble}
\algtext*{EndPEvalPrimeDouble}

\algblock{PImpNewton}{EndPImpNewton}
\algnewcommand\algorithmicPImpNewton{\textbf{\em Newton step}}
 \algnewcommand\algorithmicendPImpNewton{}
\algrenewtext{PImpNewton}[1]{\algorithmicPImpNewton\ #1}
\algrenewtext{EndPImpNewton}{\algorithmicendPImpNewton}

\algtext*{EndPImpNewton}
\begin{algorithm}[t]
\begin{algorithmic}
\State {\bf Input:} 
initial parameter $x_0 \in \R^N$, perturbation constants $\delta_n>0$, step-sizes $\{a_n, b_n\}$, operator $\Upsilon$.
\For{$n = 0,1,2,\ldots$}	
	\State Generate $\{d_n^{i}, i=1,\ldots,N\}$, independent of $\{d_m, m=0,1,\ldots,n-1\}$. 
	\State For any $i=1,\ldots,N$, $d_n^{i}$ is distributed either as an asymmetric Bernoulli (see \eqref{eq:det-proj}) or Uniform $U[-\eta,\eta]$ for some $\eta >0$ (see Remark \ref{remark:unif}). 
	\PEval
	    \State Obtain $y_n^+ = f(x_n+\delta_n d_n) + \xi_n^+$.
  \EndPEval
	    \PEvalPrime
	    \State Obtain $y_n^- = f(x_n-\delta_n d_n) + \xi_n^-$.
	    \EndPEvalPrime
	    	    \PEvalPrimeDouble
	    \State Obtain $y_n = f(x_n) + \xi_n$.
	    \EndPEvalPrimeDouble
	    \PImpNewton
		\State Update the parameter and Hessian as follows:
		\begin{align*}
		x_{n+1} = & \; x_n - a_n \Upsilon(\overline H_n)^{-1}\widehat\nabla f(x_n), \\
\overline H_n = &\; (1-b_{n})  \overline H_{n-1} + b_{n} ( \widehat H_n - \widehat \Psi_n),
\end{align*}
where $\widehat H_n$ is chosen either according to \eqref{eq:2rdsa-estimate-ber} or \eqref{eq:2rdsa-estimate-unif}.
		\EndPImpNewton
\EndFor
\State {\bf Return} $x_n.$
\end{algorithmic}
\caption{Structure of 2RDSA-IH algorithm.}
\label{alg:structure}
\end{algorithm}

Algorithm \ref{alg:structure} presents the pseudocode
\paragraph{\textbf{Function evaluations}}
Let $y_n$, $y_n^+$ and $y_n^-$ denote the function evaluations at $x_n$, $x_n+\delta_n d_n$ and $x_n - \delta_n d_n$ respectively, i.e., 
$y_n = f(x_n) + \xi_n$, $y_n^+ = f(x_n+\delta_n d_n) + \xi_n^+$ and 
$y_n^- = f(x_n-\delta_n d_n) + \xi_n^-$,
where the noise terms $\xi_n, \xi_n^+, \xi_n^-$ satisfy $\E\left[\left.\xi_n^+ + \xi_n^- - 2\xi_n \right| \F_n\right] = 0$ with $\F_n = \sigma(x_m,m\le n)$ denoting the underlying sigma-field.

Further, $\delta_n, n\geq 0$ is a sequence of diminishing positive real numbers and $d_n = (d_n^1,\ldots,d_n^N)\tr$ is the perturbation random vector at instant $n$ with $d_n^i,i=1,\ldots,N$ given in \eqref{eq:det-proj}.  

\paragraph{\textbf{Gradient estimate}}
The RDSA estimate of the gradient $\nabla f(x_n)$ is given by
\begin{align}
\label{eq:grad-ber}
\widehat\nabla f(x_n) = \frac1{1+\epsilon} d_n \left[ \dfrac{y_n^+ - y_n^-}{2\delta_n}\right],
\end{align}
where the perturbations $d_n^i$, $i=1,\ldots,N$ are i.i.d. and distributed  as follows: 
\begin{equation}
\label{eq:det-proj}
 d_n^i =
  \begin{cases}
   -1 &  \text{ w.p. } \dfrac{(1+\epsilon)}{(2+\epsilon)}, \\
   1+\epsilon &  \text{ w.p. } \dfrac{1}{(2+\epsilon)},
  \end{cases}
\end{equation}
with $\epsilon>0$ being a constant that can be chosen to be arbitrarily small.
%Note that $E d_n^i = 0$, $E (d_n^i)^2 = 1+\epsilon$ and $E (d_n^i)^4 = \dfrac{(1+\epsilon)(1+(1+\epsilon)^3)}{(2+\epsilon)}$.

\paragraph{\textbf{Hessian estimate}}

\begin{align}
\label{eq:2rdsa-estimate-ber}
&\widehat H_n = M_n \left(\dfrac{y_n^+ + y_n^- - 2 y_n}{\delta_n^2}\right), \text{ where }\\
& M_n =
\left[
\begin{array}{ccc}
\frac{1}{\kappa}\left((d_n^1)^2\!-(1+\epsilon)\right) & \cdots & \frac{1}{2(1+\epsilon)^2}d_n^1 d_n^N\\
\frac{1}{2(1+\epsilon)^2}d_n^2 d_n^1  &  \cdots & \frac{1}{2(1+\epsilon)^2}d_n^2 d_n^N\\
\cdots&\cdots&\cdots\\
\frac{1}{2(1+\epsilon)^2}d_n^N d_n^1 & \cdots &  \frac{1}{\kappa}\left((d_n^N)^2-(1+\epsilon)\right) \\
\end{array}
\right],\nonumber
\end{align}
where $\kappa = \tau \left(1- \dfrac{(1+\epsilon)^2}{\tau}\right)$ and $\tau = E (d_n^i)^4= \dfrac{(1+\epsilon)(1+(1+\epsilon)^3)}{(2+\epsilon)}$, for any $i=1,\ldots,N$. 
%%%%%% Feedback
\paragraph{\textbf{Feedback term }$\widehat \Psi_n$}
The Hessian estimate $\widehat H_n$ can be simplified as follows: 
\begin{align}
\widehat H_n & = M_n \left(\dfrac{y_n^+ + y_n^- - 2 y_n}{\delta_n^2}\right) \nonumber \\
& =  M_n \left[\left(\dfrac{f(x_n+\delta_n d_n) + f(x_n-\delta_n d_n) - 2 f(x_n)}{\delta_n^2}\right) \right.\nonumber \\&\hspace{10em}+\left. \left(\dfrac{\xi_n^+ + \xi_n^- - 2 \xi_n}{\delta_n^2}\right)\right] \nonumber \\
& = M_n \left(d_n\tr \nabla^2 f(x_n) d_n +  O(\delta_n^2) + \left(\dfrac{\xi_n^+ + \xi_n^- - 2 \xi_n}{\delta_n^2} \right)\right).\label{eq:h0}
%& = M_n \left(d_n\tr \nabla^2 f(x_n) d_n +  O(\delta_n^2) + O(\delta_n^{-2})\right). \label{eq:h0}
\end{align}
For the first term on the RHS above, note that 
\begin{align}
&\E[M_n \left(d_n\tr \nabla^2 f(x_n) d_n\right) \mid \F_n] =   \E\left[\left. M_n \times\right.\right.\nonumber\\
&\left.\left.\left(\sum\limits_{i=1}^{N-1} (d_n^i)^2 \nabla^2_{ii} f(x_n) + 2\sum\limits_{i=1}^N\sum\limits_{j=i+1}^N d_n^i d_n^j \nabla^2_{ij} f(x_n) \right)\right| \F_n\right]. \label{eq:h1}
\end{align}
In analyzing the $l$th diagonal term of the above expression, the following zero-mean term appears (see the proof of Lemma 4 in \cite{prashanth2015rdsa}):
\begin{align}
& \E\left[\left. \left([M_n]_{D}\right)_{l,l} \left(2\sum\limits_{i=1}^{N-1}\sum\limits_{j=i+1}^N d_n^i d_n^j \nabla^2_{ij} f(x_n)\right)\right| \F_{n}\right] = 0,\label{eq:diagzero}
\end{align}
where for any matrix $M$, $[M]_{D}$ refers to a matrix that retains only the diagonal entries of $M$ and replaces all the remaining  entries with zero, and $\left([M]_{D}\right)_{i,j}$ refers to the $(i,j)$th entry in $[M]_D$. We shall also use $[M]_{N}$ to refer to a matrix that retains only the off-diagonal entries of $M$, while replaces all the diagonal entries with zero.

The term on the LHS in \eqref{eq:diagzero}, denoted by $\Psi_{n}^{1}(\nabla^2 f(x_n))$, can be written in matrix form as follows: 
\begin{align}\label{eq:digmat}
\Psi_{n}^{1}(\nabla^2 f(x_n)) = [M_n]_{D}\left(d_{n}\tr \, [\nabla^2 f(x_n)]_{N} \, d_{n}\right).
\end{align}

In analyzing the off-diagonal term ($(k,l)$ where $k < l$) of \eqref{eq:h1}, the following zero-mean term appears:
\begin{align}
& \E\left[\left.\left([M_n]_{D}\right)_{k,l}   \left(\sum\limits_{i=1}^N (d_n^i)^2 \nabla^2_{ii} f(x_n)\right)\right| \F_n \right] = 0. 
\end{align}
The term on the LHS above, denoted by $\Psi_{n}^{2}(\nabla^2 f(x_n))$, can be written in matrix form as follows: 
\begin{align}\label{eq:ndmat}
\Psi_{n}^{2}(\nabla^2 f(x_n)) = [M_n]_{N}\left(d_{n}\tr \, [\nabla^2 f(x_n)]_{D} \, d_{n}\right).
\end{align}
From the foregoing, the per-iteration Hessian estimate $\widehat H_n$ can be re-written as follows:
\begin{align}
 \E\left[\left.\widehat H_n\right| \F_n\right] = & \nabla^2 f(x_n) + \E\left[\left.\Psi_{n}(\nabla^2 f(x_n))\right| \F_n\right]  +  O(\delta_n^2)\nonumber\\
&+\E\left[\left.\left(\dfrac{\xi_n^+ + \xi_n^- - 2 \xi_n}{\delta_n^2} \right)\right| \F_n\right], \label{eq:hnhat}
\end{align}
where, for any matrix $H$, 
\begin{align}
&\Psi_{n}(H) = \Psi_{n}^{1}(H) + \Psi_{n}^{2}(H)\nonumber\\
&\qquad= [M_n]_{D}\left(d_{n}\tr \, [H]_{N} \, d_{n}\right) +  [M_n]_{N}\left(d_{n}\tr \, [H]_{D} \, d_{n}\right).\label{eq:psi}
\end{align}
In the RHS of \eqref{eq:hnhat}, it is easy to see that  the second term involving $\Psi_{n}$ and the last term involving the noise are zero-mean. Moreover, since the noise is bounded by assumption, the last term in \eqref{eq:hnhat} vanishes asymptotically at the rate $O(\delta_n^{-2})$. So, the error in estimating the Hessian is due to the second term, which involves the perturbations $d_n$. This motivates the term $\widehat \Psi_n$ in the update rule \eqref{eq:e2rdsa}. 
 %and using the fact that $\E\left[\left.\xi_n^+ + \xi_n^- - 2\xi_n \right| \F_n\right] = 0$


Given that we operate in a simulation optimization setting, which implies $\nabla^2 f$ is not known, we construct the feedback term $\widehat \Psi_n$ in \eqref{eq:e2rdsa} by using $\overline H_{n-1}$ as a proxy for $\nabla^2 f$, i.e.,
\begin{align}
\widehat \Psi_n = \Psi_{n} (\overline H_{n-1}).
\end{align}

\paragraph{\textbf{Optimizing the step-sizes }$b_n$}
Unlike the feedback term, adapting the idea of optimizing the step-sizes for 2RDSA is relatively straightforward from the corresponding approach for 2SPSA in \cite{spall-jacobian}. The difference here is that there exists only one $N$-dimensional perturbation vector $d_n$ in our setting, while 2SPSA required two such vectors. This in turn implies that only the perturbation constant $\delta_n$ is needed in optimizing $b_n$.

The optimal choice for $b_n$ in \eqref{eq:2rdsa-H} is the following:
\begin{align}
\label{eq:wieghts}
b_i  = \delta_i^{4}/\sum\limits_{j=0}^{i} \delta_j^{4}.
\end{align}
The main idea behind the above choice is provided below.
From \eqref{eq:hnhat}, we can infer that
\begin{align*}
\E \parallel \widehat H_n \parallel^2 \le \dfrac{C}{\delta_n^4} \text{ for some } C<\infty. 
\end{align*} 
This is because the third term in \eqref{eq:hnhat} vanishes asymptotically, while the fourth term there dominates asymptotically. Moreover, the noise factors in the fourth term in \eqref{eq:hnhat} are bounded above due to (C9) and independent of $n$, leaving the $\delta_n^2$ term in the denominator there. 

So, the optimization problem to be solved at instant $n$ is as follows:
\begin{align}
\sum \limits_{i=0}^{n} (\tilde b_k)^2 \delta_i^{-4}, \text{ subject to} \label{eq:wn-opt}\\
\tilde b_i \geq 0 ~\forall i \text{ and }\sum \limits_{i=0}^{n} \tilde b_i = 1.
\end{align}
The optimization variable $\tilde b_i$ from the above is related to the Hessian recursion \eqref{eq:2rdsa-H} as follows:
\begin{align}
\label{eq:hess}
\overline H_n = \sum\limits_{i=0}^{n} \tilde b_k(\widehat H_i -\widehat \Psi_i).
\end{align}
The solution to \eqref{eq:wn-opt} is achieved for $\tilde b_i^* = \delta_i^{4}/\sum \limits_{j=0}^{n} \delta_j^{4}, i=1,\ldots,n$. The optimal choice $\tilde b_i^*$ can be translated to the step-sizes $b_i$, leading to \eqref{eq:wieghts}.

%%%%%%%%%%%%%%%%%%%%%%%%%%%%%%%%%%%%%%%%%%%%%%%%%%%%%%%%%%%%%%%%%%%%%%%%%%%%%%%%%%%%%%%%
\begin{remark}\textbf{\textit{(Uniform perturbations)}}
\label{remark:unif}
In \cite{prashanth2015rdsa}, the authors suggest two alternatives for the distribution of random perturbations $d_n$: the asymmetric Bernoulli, which we described earlier and uniform that we outline next.

Choose $d_n^i$, $ i=1,\ldots,N$ to be i.i.d. $U[-\eta,\eta]$ for some $\eta>0$, where $U[-\eta,\eta]$ denotes the uniform distribution on the interval $[-\eta,\eta]$.
Then, the RDSA estimate of the gradient is given by
\begin{align}
\label{eq:grad-unif}
\widehat\nabla f(x_n) = \frac3{\eta^2} d_n \left[ \dfrac{y_n^+ - y_n^-}{2\delta_n}\right].
\end{align}

The Hessian estimate in this case is given by
\begin{align}
\label{eq:2rdsa-estimate-unif}
&\widehat H_n =  M_n \left(\dfrac{y_n^+ + y_n^- - 2 y_n}{\delta_n^2}\right), \text{ where } \\
& M_n =
\dfrac{9}{2\eta^4}\left[
\begin{array}{cccc}
\frac{5}{2}\left((d_n^1)^2-\frac{\eta^2}{3}\right) & \cdots & d_n^1 d_n^N\\
d_n^2 d_n^1  &  \cdots & d_n^2 d_n^N\\
\cdots&\cdots&\cdots\\
d_n^N d_n^1 & \cdots &  \frac{5}{2}\left((d_n^N)^2-\frac{\eta^2}{3}\right) \\
\end{array}
\right].\nonumber
\end{align}

The feedback term that we propose can be easily extended to the case of uniform perturbations by using the $M_n$ as defined above instead of that for the asymmetric Bernoulli case.
\end{remark}


\section{SECOND-ORDER SPSA WITH IMPROVED HESSIAN ESTIMATION (2SPSA-IH)}

\paragraph{\textbf{Function evaluations}}
Let $y_n$, $y_n^+$ and $y_n^-$ denote the function evaluations at $x_n$, $x_n+\delta_n \Delta_n+\delta_n \widehat \Delta_n$ and $x_n -\delta_n\Delta_n-\delta_n\widehat\Delta_n$ respectively, i.e., 
$y_n = f(x_n) + \xi_n$, $y_n^+ = f(x_n+\delta_n \Delta_n +\delta_n \widehat\Delta_n) + \xi_n^+$ and 
$y_n^- = f(x_n -\delta_n\Delta_n-\delta_n\widehat\Delta_n) + \xi_n^-$,
where the noise terms $\xi_n, \xi_n^+, \xi_n^-$ satisfy $\E\left[\left.\xi_n^+ + \xi_n^- - 2\xi_n \right| \F_n\right] = 0$ with $\F_n = \sigma(x_m,m\le n)$ denoting the underlying sigma-field.

Further, $\delta_n, n\geq 0$ is a sequence of diminishing positive real numbers and $\Delta_n = (\Delta_n^1,\ldots,\Delta_n^N)\tr$ , $\widehat\Delta_n = (\widehat\Delta_n^1,\ldots,\widehat\Delta_n^N)\tr$ are the perturbation random vector at instant $n$ with $\Delta_n^i,i=1,\ldots,N$, and $\widehat\Delta_n^i,i=1,\ldots,N$ as Bernoulli $\pm$1 random variables.  

\paragraph{\textbf{Gradient estimate}}
The SPSA estimate of the gradient $\nabla f(x_n)$ is given by
\begin{align}
\label{eq:grad-spsa}
\widehat\nabla_{(i)} f(x_n) =  \left[ \dfrac{y_n^+ - y_n^-}{2\delta_n \Delta_n^{(i)}}\right],
\end{align}
where the perturbations $\Delta_n^i$, $i=1,\ldots,N$ are i.i.d. and Bernoulli $\pm$1 random variables.

%Note that $E d_n^i = 0$, $E (d_n^i)^2 = 1+\epsilon$ and $E (d_n^i)^4 = \dfrac{(1+\epsilon)(1+(1+\epsilon)^3)}{(2+\epsilon)}$.

\paragraph{\textbf{Hessian estimate}}
The $i,j$th entry of the Hessian estimate in this case is given by
\begin{align}
\label{eq:2spsa-ber}
&\left(\widehat H_n\right)_{ij} =  \left(\dfrac{y_n^+ + y_n^- - 2 y_n}{\delta_n^2 \Delta_n^{(i)} \widehat \Delta_n^{(j)}}\right), 
\end{align}
%%%%%% Feedback
\paragraph{\textbf{Feedback term }$\widehat \Psi_n$}
The $i,j$th term of the Hessian estimate $\widehat H_n$ can be simplified as follows: 
\begin{align}
\left(\widehat H_n \right)_{ij} & =  \left(\dfrac{y_n^+ + y_n^- - 2 y_n}{\delta_n^2 \Delta_n^{(i)} \widehat \Delta_n^{(j)}}\right) \nonumber \\
%& =  \left[\left(\dfrac{f(x_n+\delta_n \Delta_n+\delta_n \widehat \Delta_n) + f(x_n -\delta_n\Delta_n-\delta_n\widehat\Delta_n) - 2 f(x_n)}{\delta_n^2 \Delta_n^{(i)} \widehat \Delta_n^{(j)}}\right) \right.\nonumber \\&\hspace{10em}+\left. \left(\dfrac{\xi_n^+ + \xi_n^- - 2 \xi_n}{\delta_n^2}\right)\right] \nonumber \\
& = \left( \frac{(\Delta_n+ \widehat \Delta_n)\tr \nabla^2 f(x_n) (\Delta_n+ \widehat \Delta_n)}{\Delta_n^{(i)} \widehat \Delta_n^{(j)}} +  O(\delta_n^2) + \right.\nonumber \\&\hspace{10em}+\left.\left(\dfrac{\xi_n^+ + \xi_n^- - 2 \xi_n}{\delta_n^2 \Delta_n^{(i)} \widehat \Delta_n^{(j)}} \right)\right).\label{eq:h3}
%& = M_n \left(d_n\tr \nabla^2 f(x_n) d_n +  O(\delta_n^2) + O(\delta_n^{-2})\right). \label{eq:h0}
\end{align}
After expanding the first term on the RHS above, we get
\begin{align}
\sum\limits_{l=1}^N\sum\limits_{m=1}^N \frac{\Delta_n^l \nabla^2_{lm} f(x_n) \Delta_n^m}{\Delta_n^i \widehat\Delta_n^j} &+ 2 \sum\limits_{l=1}^N\sum\limits_{m=1}^N \frac{\Delta_n^l \nabla^2_{lm} f(x_n) \widehat\Delta_n^m}{\Delta_n^i \widehat\Delta_n^j} \nonumber \\ & + \sum\limits_{l=1}^N\sum\limits_{m=1}^N \frac{\widehat\Delta_n^l \nabla^2_{lm} f(x_n)\widehat \Delta_n^m}{\Delta_n^i \widehat\Delta_n^j}\label{eq:2spsaexp}
\end{align}
It is easy to see that 
\begin{align}
&\E\left[\left.\sum\limits_{l=1}^N\sum\limits_{m=1}^N \frac{\Delta_n^l \nabla^2_{lm} f(x_n) \Delta_n^m}{\Delta_n^i \widehat\Delta_n^j} \right | \F_n\right] = 0 \nonumber\\ &\E\left[\left.\sum\limits_{l=1}^N\sum\limits_{m=1}^N \frac{\widehat\Delta_n^l \nabla^2_{lm} f(x_n) \widehat\Delta_n^m}{\Delta_n^i \widehat\Delta_n^j }\right | \F_n\right] = 0 \nonumber\\ &\E\left[\left. \sum\limits_{l=1}^N\sum\limits_{m=1}^N \frac{\Delta_n^l \nabla^2_{lm} f(x_n) \widehat\Delta_n^m}{\Delta_n^i \widehat\Delta_n^j} \right | \F_n\right] = \nabla^2_{ij} f(x_n) \hspace{0.3cm}a.s..
\end{align}
From the above equation we know that first and second terms are mean zero terms, they can be represented in matrix form as below
\begin{align}
\Psi_n^1(\nabla^2 f(x_n)) = M_n \left[\Delta_n\tr \nabla^2 f(x_n) \Delta_n + \widehat\Delta_n\tr \nabla^2 f(x_n) \widehat\Delta_n\right]
\end{align}
where $M_n = [1/\Delta_n^1, \ldots ,1/\Delta_n^N]\tr [1/\Delta_n^1, \ldots ,1/\Delta_n^N]$.
Now consider the second term in the \eqref{eq:2spsaexp}. It can be written as following 
\begin{align}
\nabla^2_{ij} f(x_n) + \frac{1}{\Delta_n^i \widehat\Delta_n^j} \mathop{\sum_{l=1}^N \sum_{m=1}^{N}}_{lm \neq ij} \Delta_n^l \nabla^2_{lm} f(x_n) \widehat\Delta_n^m
\end{align}
Now in the above equation the second term is mean zero term and it can be represented as follows
\begin{align}
\Psi_n^2(\nabla^2 f(x_n)) = \widehat N_n\tr \nabla^2 f(x_n) N_n  &+ \widehat N_n\tr \nabla^2 f(x_n) \nonumber\\ &+ \nabla^2 f(x_n) N_n
\end{align}
where $N_n,\widehat N_n$ are defined as follows $N_n = \Delta_n \left[\frac{1}{\Delta_n^1},\ldots,\frac{1}{\Delta_n^N}\right] - I_N$ and $\widehat N_n = \widehat \Delta_n \left [\frac{1}{\widehat\Delta_n^1},\ldots,\frac{1}{\widehat \Delta_n^N}\right] - I_N$. Where $I_N$ is $N \times N$ identity matrix.
From the foregoing, the per-iteration Hessian estimate $\widehat H_n$ can be re-written as follows:
\begin{align}
H_n = & \nabla^2 f(x_n) + \Psi_{n}(\nabla^2 f(x_n)) +  O(\delta_n^2) + O(\delta_n^{-2})  \label{eq:spsahnhat}
\end{align}
where, for any matrix $H$, 
\begin{align}
\Psi_{n}(H) &=  \Psi_{n}^{1}(H) + \Psi_{n}^{2}(H)\nonumber\\
&=  M_n \left[\Delta_n\tr H \Delta_n + \widehat\Delta_n\tr H \widehat\Delta_n\right] + \widehat N_n\tr H N_n  \nonumber\\ & \hspace{3.5cm}+ \widehat N_n\tr H + H N_n.\label{eq:spsapsi}
\end{align}



Given that we operate in a simulation optimization setting, which implies $\nabla^2 f$ is not known, we construct the feedback term $\widehat \Psi_n$ in \eqref{eq:e2rdsa} by using $\overline H_{n-1}$ as a proxy for $\nabla^2 f$, i.e.,
\begin{align}
\widehat \Psi_n = \Psi_{n} (\overline H_{n-1}).
\end{align}

\paragraph{\textbf{Optimizing the step-sizes }$b_n$}

The optimal choice for $b_n$ in \eqref{eq:2rdsa-H} \textbf {for this set up} is the following:
\begin{align}
\label{eq:spsawieghts}
b_i  = \delta_i^{4}/\sum\limits_{j=0}^{i} \delta_j^{4}.
\end{align}
The main idea behind the above choice is provided below.
From \eqref{eq:spsahnhat}, we can infer that
\begin{align*}
\E \parallel \widehat H_n \parallel^2 \le \dfrac{C}{\delta_n^4} \text{ for some } C<\infty. 
\end{align*} 
This is because the third term in \eqref{eq:spsahnhat} vanishes asymptotically, while the fourth term there dominates asymptotically. Moreover, the noise factors in the fourth term in \eqref{eq:spsahnhat} are bounded above due to \textbf{(C9) equivalent } and independent of $n$, leaving the $\delta_n^2$ term in the denominator there. 

So, the optimization problem to be solved at instant $n$ is as follows:
\begin{align}
\sum \limits_{i=0}^{n} (\tilde b_k)^2 \delta_i^{-4}, \text{ subject to} \label{eq:spsawn-opt}\\
\tilde b_i \geq 0 ~\forall i \text{ and }\sum \limits_{i=0}^{n} \tilde b_i = 1.
\end{align}
The optimization variable $\tilde b_i$ from the above is related to the Hessian recursion \eqref{eq:2rdsa-H} as follows:
\begin{align}
\label{eq:spsahess}
\overline H_n = \sum\limits_{i=0}^{n} \tilde b_k(\widehat H_i -\widehat \Psi_i).
\end{align}
The solution to \eqref{eq:spsawn-opt} is achieved for $\tilde b_i^* = \delta_i^{4}/\sum \limits_{j=0}^{n} \delta_j^{4}, i=1,\ldots,n$. The optimal choice $\tilde b_i^*$ can be translated to the step-sizes $b_i$, leading to \eqref{eq:spsawieghts}.







\section{CONVERGENCE ANALYSIS FOR 2RDSA-IH}
\label{sec:2rdsa-results}
We make the same assumptions as those used in the analysis of \cite{prashanth2015rdsa}, with a few minor alterations. The assumptions are listed below:
\begin{enumerate}[label=(\textbf{C\arabic*})]
\item  The function
$f$ is four-times differentiable\footnote{Here $\nabla^4 f(x) = \dfrac{\partial^4 f (x)}{\partial x\tr \partial x\tr \partial x\tr \partial x\tr}$ denotes the fourth derivate of $f$ at $x$ and $\nabla^4_{i_1 i_2 i_3 i_4} f(x)$ denotes the $(i_1 i_2 i_3 i_4)$th entry of $\nabla^4 f(x)$, for $i_1, i_2, i_3,i_4=1,\ldots, N$.} with $\left|\nabla^4_{i_1 i_2 i_3 i_4} f(x) \right| < \infty$, for $i_1, i_2, i_3,i_4=1,\ldots, N$ and for all $x\in \R^N$. 

%\item  For some $\rho>0$  and almost all $x_n$, the function $f$ is four-times differentiable with a uniformly (in $n$) bounded fourth derivative for all $x$ such that $\left\| x_n - x\right\| \le \rho$. 

\item For each $n$ and all $x$, there exists a $\rho>0$ not dependent on $n$ and $x$, such that $(x-x^*)\tr \bar f_n(x) \ge \rho \left\| x_n - x\right\|$, where $\bar f_n(x) = \Upsilon(\overline H_n)^{-1} \nabla f(x)$.

\item $\{\xi_n, \xi_n^+,\xi_n^-, n=1,2,\ldots\}$ satisfy $\E\left[\left. \xi_n^+ + \xi_n^- - 2 \xi_n \right| \F_n\right] = 0$, for all $n$, where $\mathcal{F}_n = \sigma(x_m,m\le n)$ denotes the underlying sigma-field.. 

\item $\{d_n^i, i=1,\ldots,N, n=1,2,\ldots\}$ are i.i.d. and independent of $\F_n$.

\item  The step-sizes $a_n$ and perturbation constants $\delta_n$ are positive, for all $n$ and satisfy
$$\hspace{-2em} a_n, \delta_n \rightarrow 0\text{ as } n \rightarrow \infty, 
\sum_n a_n=\infty \text{ and } \sum_n \left(\frac{a_n}{\delta_n}\right)^2 <\infty.$$

\item For each $i=1,\ldots,N$ and any $\rho>0$, 
$P(\{ \bar f_{ni} (x_n) \ge 0 \text{ i.o}\} \cap \{ \bar f_{ni} (x_n) < 0 \text{ i.o}\} \mid \{ |x_{ni} - x^*_i| \ge \rho\quad \forall n\}) =0.$

\item The operator $\Upsilon$ satisfies $\delta_n^2 \Upsilon(H_n)^{-1} \rightarrow 0$ a.s. and  $E(\left\| \Upsilon(H_n)^{-1}\right\|^{2+\zeta}) \le \rho$ for some $\zeta, \rho>0$.

\item For any $\tau >0$ and nonempty $S \subseteq \{1,\ldots,N\}$, there exists a $\rho'(\tau,S)>\tau$ such that 
$$ \limsup_{n\rightarrow \infty} \left| \dfrac{\sum_{i \notin S} (x-x^*)_i \bar f_{ni}(x)}{\sum_{i \in S} (x-x^*)_i \bar f_{ni}(x)}               \right| < 1 \text{ a.s.}$$
for all $|(x-x^*)_i| < \tau$ when $i \notin S$ and $|(x-x^*)_i| \ge \rho'(\tau,S)$ when $i\in S$.
\item For some $\alpha_0, \alpha_1>0$ and for all $n$, $\E {\xi_n}^{2} \le \alpha_0$, $\E {\xi_n^{\pm}}^{2} \le \alpha_0$, $\E f(x_n)^{2} \le \alpha_1$,  $\E f(x_n\pm \delta_n d_n)^{2} \le \alpha_1$ and $\E \left(\left\| \Upsilon(\overline H_n \right\|^2 \mid \F_n\right) \le \alpha_1$. 
\item  $\delta_n = \frac{\delta_0}{(n+1)^{\varsigma}}$, where $\delta_0 > 0$ and $0 < \varsigma \le 1/8$.
\end{enumerate}
The reader is referred to Section II-B of \cite{prashanth2015rdsa} for a detailed discussion of the above assumptions. We remark here that (C1)-(C8) are identical to that in \cite{prashanth2015rdsa}, while (C9) and (C10) introduce minor additional requirements on $\left\| \Upsilon(\overline H_n \right\|^2$ and $\delta_n$, respectively and these are inspired by \cite{spall-jacobian}.

\begin{lemma}(\textbf{Bias in Hessian estimate})
\label{lemma:2rdsa-bias}
Under (C1)-(C10), with $\widehat H_n$ defined according to either \eqref{eq:2rdsa-estimate-unif} or \eqref{eq:2rdsa-estimate-ber}, we have a.s. that\footnote{Here $\widehat H_n(i,j)$ and $\nabla^2_{ij}f(\cdot)$ denote the $(i,j)$th entry in the Hessian estimate $\widehat H_n$ and the true Hessian $\nabla^2 f(\cdot)$, respectively.}, for $i,j = 1,\ldots,N$,
\begin{align}
\left|\E\left[
\left. \widehat H_n(i,j) \right| \F_n \right] - \nabla^2_{ij} f(x_n)\right| = O(\delta_n^2).
\end{align} 
\end{lemma}
\begin{proof}
See Lemma 4 in \cite{prashanth2015rdsa}.
\end{proof}

\begin{theorem}(\textbf{Strong Convergence of Hessian})
\label{thm:2rdsa-H}
Under (C1)-(C10), we have that 
$$\overline H_n \rightarrow \nabla^2 f(x^*) \text{ a.s. as } n\rightarrow \infty.$$ 
In the above, $\overline H_n$ is updated according to \eqref{eq:2rdsa-H} and $\widehat H_n$ defined according to either \eqref{eq:2rdsa-estimate-ber} or \eqref{eq:2rdsa-estimate-unif}. 
\end{theorem}
\begin{proof}

For proving the main claim regarding $\overline H_n$, we closely follow the approach used to prove a corresponding result for 2SPSA (see Theorem 1 in \cite{spall-jacobian}). 
The first step is to prove the following:
\begin{align}
\sum_{k=0}^n \dfrac{\delta_k^4 \left(\widehat H_k - \widehat \Psi_k - \E(\widehat H_k \mid \F_k)\right)}{\sum_{i=0}^n \delta_i^4} \rightarrow 0.
\label{eq:step1}
\end{align}

By a completely parallel argument to that used in the proof of Theorem 1 in \cite{spall-jacobian}, we obtain: For any $i,j = 1,\ldots,N$,
\begin{align*}
\E \left[\left((\widehat {H_k})_{i,j} - (\widehat {\Psi_k})_{i,j} - \E((\widehat {H_k})_{i,j} \mid \F_k)\right)^2\right] = O(\delta_k^{-4}).
\end{align*}
Now \eqref{eq:step1} follows by an application of Kronecker's Lemma along with the martingale convergence theorem (see Theorem 6.2.1 of \cite{lahaprobability}).

From Lemma \ref{lemma:2rdsa-bias}, we have 
$$ \E[ \widehat H_k \mid \F_k] = \nabla^2 f(x_n) + O(\delta_n^2) \text{ a.s.}$$
Since the Hessian is continuous near $x_n$ and $x_n$ converges almost surely to $x^*$, we have
\footnote{Since assumptions (C1)-(C10) here are similar to that in \cite{prashanth2015rdsa}, Theorem 5 of \cite{prashanth2015rdsa} holds here as well. In particular, this implies almost sure convergence of $x_n$ to $x^*$.}
\begin{align*}
\sum_{k=0}^n \dfrac{\delta_k^4 \left(\E(\widehat H_k \mid \F_k)\right)}{\sum_{i=0}^n \delta_i^4} 
=&\sum_{k=0}^n \dfrac{\delta_k^4 \left(\nabla^2 f(x_n) + O(\delta_n^2)\right)}{\sum_{i=0}^n \delta_i^4}\\
=&\sum_{k=0}^n \dfrac{\delta_k^4 \left(\nabla^2 f(x^*) + o(1)\right)}{\sum_{i=0}^n \delta_i^4}\\
&\rightarrow \nabla^2 f(x^*) \text{ a.s. as } n \rightarrow \infty.
\end{align*}
The last step above follows from Toeplitz Lemma (see p. 89 of \cite{lahaprobability}) after observing that $\sum_{i=0}^n \delta_i^4 \rightarrow \infty$ due to (C10). 
The main claim now follows since 
$$ \overline H_n = \sum_{k=0}^n \dfrac{\delta_k^4 \left(\widehat H_k - \Psi_k \right)}{\sum_{i=0}^n \delta_i^4}.$$
\end{proof}



\section{COVERGENCE ANALYSIS FOR 2SPSA-IH}
\label{sec:2spsa-results}
We make the same assumptions as those used in the analysis of \cite{prashanth2015rdsa}, with a few minor alterations. The assumptions are listed below:
\begin{enumerate}[label=(\textbf{C\arabic*}),resume]
\item  The function
$f$ is four-times differentiable\footnote{Here $\nabla^4 f(x) = \dfrac{\partial^4 f (x)}{\partial x\tr \partial x\tr \partial x\tr \partial x\tr}$ denotes the fourth derivate of $f$ at $x$ and $\nabla^4_{i_1 i_2 i_3 i_4} f(x)$ denotes the $(i_1 i_2 i_3 i_4)$th entry of $\nabla^4 f(x)$, for $i_1, i_2, i_3,i_4=1,\ldots, N$.} with $\left|\nabla^4_{i_1 i_2 i_3 i_4} f(x) \right| < \infty$, for $i_1, i_2, i_3,i_4=1,\ldots, N$ and for all $x\in \R^N$. 

%\item  For some $\rho>0$  and almost all $x_n$, the function $f$ is four-times differentiable with a uniformly (in $n$) bounded fourth derivative for all $x$ such that $\left\| x_n - x\right\| \le \rho$. 

\item For each $n$ and all $x$, there exists a $\rho>0$ not dependent on $n$ and $x$, such that $(x-x^*)\tr \bar f_n(x) \ge \rho \left\| x_n - x\right\|$, where $\bar f_n(x) = \Upsilon(\overline H_n)^{-1} \nabla f(x)$.

\item $\{\xi_n, \xi_n^+,\xi_n^-, n=1,2,\ldots\}$ are such that, for all $n$, $\E\left[\left. \xi_n^+ + \xi_n^- - 2 \xi_n \right| \F_n\right] = 0$, where $\mathcal{F}_n = \sigma(x_m,m\le n)$ denotes the underlying sigma-field.. 

\item $\{\Delta_n^i,\widehat\Delta_n^i, i=1,\ldots,N, n=1,2,\ldots\}$ are i.i.d. and independent of $\F_n$.

\item  The step-sizes $a_n$ and perturbation constants $\delta_n$ are positive, for all $n$ and satisfy
$$\hspace{-1.7em} a_n, \delta_n \rightarrow 0\text{ as } n \rightarrow \infty, 
\sum_n a_n=\infty \text{ and } \sum_n \left(\frac{a_n}{\delta_n}\right)^2 <\infty.$$

\item For each $i=1,\ldots,N$ and any $\rho>0$, 
$P(\{ \bar f_{ni} (x_n) \ge 0 \text{ i.o}\} \cap \{ \bar f_{ni} (x_n) < 0 \text{ i.o}\} \mid \{ |x_{ni} - x^*_i| \ge \rho\quad \forall n\}) =0.$

\item The operator $\Upsilon$ satisfies $\delta_n^2 \Upsilon(H_n)^{-1} \rightarrow 0$ a.s. and  $E(\left\| \Upsilon(H_n)^{-1}\right\|^{2+\zeta}) \le \rho$ for some $\zeta, \rho>0$.

\item For any $\tau >0$ and nonempty $S \subseteq \{1,\ldots,N\}$, there exists a $\rho'(\tau,S)>\tau$ such that 
$$ \limsup_{n\rightarrow \infty} \left| \dfrac{\sum_{i \notin S} (x-x^*)_i \bar f_{ni}(x)}{\sum_{i \in S} (x-x^*)_i \bar f_{ni}(x)}               \right| < 1 \text{ a.s.}$$
for all $|(x-x^*)_i| < \tau$ when $i \notin S$ and $|(x-x^*)_i| \ge \rho'(\tau,S)$ when $i\in S$.
\item For some $\alpha_0, \alpha_1>0$ and for all $n$, $\E {\xi_n}^{2} \le \alpha_0$, $\E {\xi_n^{\pm}}^{2} \le \alpha_0$, $\E f(x_n)^{2} \le \alpha_1$,  $\E f(x_n + \delta_n \Delta_n + \delta_n \widehat \Delta_n)^{2},\E f(x_n - \delta_n \Delta_n - \delta_n \widehat \Delta_n)^{2} \le \alpha_1$ and $\E \left(\left\| \Upsilon(\overline H_n) \right\|^2 \mid \F_n\right) \le \alpha_1$. 
\item  $\delta_n = \frac{\delta_0}{(n+1)^{\varsigma}}$, where $\delta_0 > 0$ and $0 < \varsigma \le 1/8$.
\end{enumerate}
The reader is referred to Section II-B of \cite{prashanth2015rdsa} for a detailed discussion of the above assumptions. We remark here that (C13)-(C20) are identical to that in \cite{prashanth2015rdsa}, while (C21) and (C22) introduce minor additional requirements on $\left\| \Upsilon(\overline H_n \right\|^2$ and $\delta_n$, respectively and these are inspired by \cite{spall-jacobian}.

\begin{lemma}(\textbf{2SPSA-IH Bias in Hessian estimate})
\label{lemma:2spsa-bias}
Under \textbf{(C13)-(C22) equivalent}, with $\widehat H_n$ defined according to  \eqref{eq:2spsa-ber} , we have a.s. that\footnote{Here $\widehat H_n(i,j)$ and $\nabla^2_{ij}f(\cdot)$ denote the $(i,j)$th entry in the Hessian estimate $\widehat H_n$ and the true Hessian $\nabla^2 f(\cdot)$, respectively.}, for $i,j = 1,\ldots,N$,
\begin{align}
\left|\E\left[
\left. \widehat H_n(i,j) \right| \F_n \right] - \nabla^2_{ij} f(x_n)\right| = O(\delta_n^2).
\end{align} 
\end{lemma}
\begin{proof}
See Proposition 4.2 in \cite{bhatnagar2015simultaneous}.
\end{proof}

\begin{theorem}(\textbf{2SPSA-IH Strong Convergence of Hessian})
\label{thm:2spsa-H}
Under (C13)-(C22), we have that 
$$\overline H_n \rightarrow \nabla^2 f(x^*) \text{ a.s. as } n\rightarrow \infty.$$ 
In the above, $\overline H_n$ is updated according to \eqref{eq:2rdsa-H}. $\widehat H_n$ defined according to \eqref{eq:2spsa-ber} and the step-sizes $b_n$ are chosen as suggested in \eqref{eq:spsawieghts}. 
\end{theorem}
\begin{proof}
This proof is exactly similar to the proof of the theorem \ref{thm:2rdsa-H}
\end{proof}

We next present a convergence rate result for the special case of a quadratic objective function under the following additional assumptions:
\begin{enumerate}[label=(\textbf{C\arabic*}),resume]
\item  $f$ is quadratic and $\nabla^2 f(x^*) > 0$. 
\item The operator $\Upsilon$ is chosen such that $\E(\parallel \Upsilon(\overline H_n) - \overline H_n\parallel^2) = o(e^{-2wn^{1-r}/(1-r)})$ and $\parallel \Upsilon(H) - H \parallel^2 / (1+\parallel H \parallel^2)$ is uniformly bounded.
\end{enumerate}
\begin{theorem}(\textbf{2SPSA-IH Quadratic case - Convergence rate})
\label{thm:spsaquad-bound}
Assume (C16), (C22), (C23) and (C24) and also that the setting is noise-free. 
Let $b_n = b_0/n^r$, $n=1,2,\ldots,k$, where $1/2 < r < 1$ and $0 < b_0 \leq 1$. For notational simplicity, let $H^*=\nabla^2 f(x^*)$. Letting $\Lambda_k = \overline H_k - H^*$, we have 
\begin{align}
\text{trace}[\E (\Lambda_n \tr \Lambda_n)] = O(e^{-2b_0n^{1-r} / {1-r}}).
\label{eq:spsaquad-bigo}
\end{align}
\end{theorem}
\begin{proof}
This proof directly follows from the proof of theorem 3 of \cite{spall-jacobian}
\end{proof}



\subsection{Gradient estimate}

\subsection{Main results}

\section{Numerical Experiments}





\section{Conclusions}

\section*{Appendix}

%%%%%%%%%%%%%%%%%%%%%%%%%%%%%%%%%%%%%%%%%%%%%%%%%%%%%%%%%%%%%%%%%%%%%%%%%%%%%%%%%%%%%%%%%%%%%%%%%%%%%%%%%%%%%%%%%%%%%%%%%%%%%%%%%%%%%%
\bibliographystyle{IEEEtran}
\bibliography{reference}
\end{document}





