%%%%%%%%%%%%%%%%%%%%%%%%%%%%%%%%%%%%%%%%%%%%%%%%%%%%%%%%%%%%%%%%%%%%%%%%%%%%%%%%
%2345678901234567890123456789012345678901234567890123456789012345678901234567890
%        1         2         3         4         5         6         7         8

\documentclass[letterpaper, 10 pt, conference]{ieeeconf}  % Comment this line out if you need a4paper

%\documentclass[a4paper, 10pt, conference]{ieeeconf}      % Use this line for a4 paper

\IEEEoverridecommandlockouts                              % This command is only needed if 
                                                          % you want to use the \thanks command

\overrideIEEEmargins                                      % Needed to meet printer requirements.

% See the \addtolength command later in the file to balance the column lengths
% on the last page of the document

% The following packages can be found on http:\\www.ctan.org
%\usepackage{graphics} % for pdf, bitmapped graphics files
%\usepackage{epsfig} % for postscript graphics files
%\usepackage{mathptmx} % assumes new font selection scheme installed
%\usepackage{times} % assumes new font selection scheme installed
%\usepackage{amsmath} % assumes amsmath package installed
%\usepackage{amssymb}  % assumes amsmath package installed

\usepackage{macros}


\title{\LARGE \bf
Enhanced Adaptive Random direction Stochastic approximation
}



\author{D. Sai Koti Reddy$^\dagger$, Prashanth L A$^\sharp$, Shalabh Bhatnagar$^\ddag$
\thanks{
$^\dagger$ Department of Computer Science and Automation,
Indian Institute of Science, Bangalore,
E-Mail: danda.reddy@csa.iisc.ernet.in}
\thanks{
$^\sharp$ Institute for Systems Research, University of Maryland, College Park, Maryland,
E-Mail: prashanth@isr.umd.edu.
}
\thanks{
$^\ddag$ Department of Computer Science and Automation,
Indian Institute of Science, Bangalore,
E-Mail: shalabh@csa.iisc.ernet.in
}
}



\begin{document}


\nocite{*}
\maketitle
\thispagestyle{empty}
\pagestyle{empty}


%%%%%%%%%%%%%%%%%%%%%%%%%%%%%%%%%%%%%%%%%%%%%%%%%%%%%%%%%%%%%%%%%%%%%%%%%%%%%%%%
\begin{abstract}



\end{abstract}


%%%%%%%%%%%%%%%%%%%%%%%%%%%%%%%%%%%%%%%%%%%%%%%%%%%%%%%%%%%%%%%%%%%%%%%%%%%%%%%%
\section{INTRODUCTION}
\section{Gradient estimate}\label{sec:grad}
The gradient estimate in this paper is same as that of in \cite{prashanth2015rdsa}.
\subsection*{\textbf{Uniform perturbations}}
Choose $d_n^i$, $ i=1,\ldots,N$ to be i.i.d. $U[-\eta,\eta]$ for some $\eta>0$, where $U[-\eta,\eta]$ denotes the uniform distribution on the interval $[-\eta,\eta]$.
The RDSA estimate of the gradient is given by
\begin{align}
\label{eq:grad-unif}
\widehat\nabla f(x_n) = \frac3{\eta^2} d_n \left[ \dfrac{y_n^+ - y_n^-}{2\delta_n}\right].
\end{align}

\subsection*{\textbf{Asymmetric Bernoulli perturbations}}
Choose $d_n^i$, $i=1,\ldots,N$, i.i.d. as follows: 
\begin{equation}
\label{eq:det-proj}
 d_n^i =
  \begin{cases}
   -1 &  \text{ w.p. } \dfrac{(1+\epsilon)}{(2+\epsilon)}, \\
   1+\epsilon &  \text{ w.p. } \dfrac{1}{(2+\epsilon)},
  \end{cases}
\end{equation}
where $\epsilon>0$ is a constant that can be chosen to be arbitrarily small.
Note that $E d_n^i = 0$, $E (d_n^i)^2 = 1+\epsilon$ and $E (d_n^i)^4 = \dfrac{(1+\epsilon)(1+(1+\epsilon)^3)}{(2+\epsilon)}$.
Then, the RDSA estimate of the gradient is given by
\begin{align}
\label{eq:grad-ber}
\widehat\nabla f(x_n) = \frac1{1+\epsilon} d_n \left[ \dfrac{y_n^+ - y_n^-}{2\delta_n}\right].
\end{align}



\section{enhanced second-order rdsa} \label{sec:e2rdsa}
The iterative scheme for the proposed second-ordered adaptive search algorithms is of the following form :
\begin{align}
\label{eq:e2rdsa}
x_{n+1} = x_n - a_n \Upsilon(\overline H_n)^{-1}\widehat\nabla f(x_n), \\
\overline H_n = (1-w_{n})  \overline H_{n-1} + w_{n} ( \widehat H_n - \widehat \Psi_n).\label{eq:2rdsa-H}
\end{align}
In the above, 
\begin{itemize}
 \item $\widehat\nabla f(x_n)$ is the estimate of $\nabla f(x_n)$ and this corresponds to  the uniform  and   asymmetric Bernoulli variant gradient estimates proposed in \cite{prashanth2015rdsa}.
 \item $\widehat H_n$ is an estimate of the true Hessian ${\nabla}^2 f(\cdot)$, with $\widehat H_0 = I$. 
 \item $\overline H_n$ is a smoothed version of $\widehat H_n$, which is crucial to ensure convergence. 
 \item $\Upsilon$ is an operator that projects a matrix onto the set of positive definite matrices. Update \eqref{eq:2rdsa-H} does not necessarily ensure that $\overline H_n$ is invertible and without $\Upsilon$, the parameter update \eqref{eq:e2rdsa} may not move along a descent direction - see conditions (C7) and (C12) in Section \ref{sec:2rdsa-results} below for the precise requirements on the matrix projection operator.
 \item $\widehat \Psi_n$ term is feedback term, which improves the hessian estimate.
 \item $w_{n}$ are weights, which optimizes the effects of variance.
\end{itemize}


\subsection{Hessian estimate}

\subsubsection*{\textbf{Uniform perturbations}}

\begin{align}
\label{eq:2rdsa-estimate-unif}
&\widehat H_n = \dfrac{9}{2\eta^4} M_n \left(\dfrac{y_n^+ + y_n^- - 2 y_n}{\delta_n^2}\right), \text{ where } \\
& M_n =
\left[
\begin{array}{cccc}
\frac{5}{2}\left((d_n^1)^2-\frac{\eta^2}{3}\right) & \cdots & d_n^1 d_n^N\\
d_n^2 d_n^1  &  \cdots & d_n^2 d_n^N\\
d_n^N d_n^1 & \cdots &  \frac{5}{2}\left((d_n^N)^2-\frac{\eta^2}{3}\right) \\
\end{array}
\right].\nonumber
\end{align}
%Henceforth, we shall refer to algorithm \eqref{eq:2rdsa}--\eqref{eq:2rdsa-H} with Hessian estimate \eqref{eq:2rdsa-estimate} as 2RDSA.

\subsubsection*{\textbf{Asymmetric Bernoulli perturbations}}

\begin{align}
\label{eq:2rdsa-estimate-ber}
&\widehat H_n = M_n \left(\dfrac{y_n^+ + y_n^- - 2 y_n}{\delta_n^2}\right), \text{ where }\\
& M_n =
\left[
\begin{array}{cccc}
\frac{1}{\kappa}\left((d_n^1)^2-(1+\epsilon)\right) & \cdots & \frac{1}{2(1+\epsilon)^2}d_n^1 d_n^N\\
\frac{1}{2(1+\epsilon)^2}d_n^2 d_n^1  &  \cdots & \frac{1}{2(1+\epsilon)^2}d_n^2 d_n^N\\
\frac{1}{2(1+\epsilon)^2}d_n^N d_n^1 & \cdots &  \frac{1}{\kappa}\left((d_n^N)^2-(1+\epsilon)\right) \\
\end{array}
\right],\nonumber
\end{align}
where $\kappa = E (d_n^i)^4 \left(1- \dfrac{(1+\epsilon)^2}{E (d_n^i)^4 }\right)$, with $E (d_n^i)^4= \dfrac{(1+\epsilon)(1+(1+\epsilon)^3)}{(2+\epsilon)}$, $i=1,\ldots,N$.

\subsection{Main results}
\label{sec:2rdsa-results}
 $\mathcal{F}_n = \sigma(x_m,m\le n)$ denotes the underlying sigma-field. 
We make the following assumptions that are similar to those in \cite{spall_adaptive}:
\begin{enumerate}[label=(\textbf{C\arabic*})]
\item  The function
$f$ is four-times differentiable\footnote{Here $\nabla^4 f(x) = \dfrac{\partial^4 f (x)}{\partial x\tr \partial x\tr \partial x\tr \partial x\tr}$ denotes the fourth derivate of $f$ at $x$ and $\nabla^4_{i_1 i_2 i_3 i_4} f(x)$ denotes the $(i_1 i_2 i_3 i_4)$th entry of $\nabla^4 f(x)$, for $i_1, i_2, i_3,i_4=1,\ldots, N$.} with $\left|\nabla^4_{i_1 i_2 i_3 i_4} f(x) \right| < \infty$, for $i_1, i_2, i_3,i_4=1,\ldots, N$ and for all $x\in \R^N$. 

%\item  For some $\rho>0$  and almost all $x_n$, the function $f$ is four-times differentiable with a uniformly (in $n$) bounded fourth derivative for all $x$ such that $\left\| x_n - x\right\| \le \rho$. 

\item For each $n$ and all $x$, there exists a $\rho>0$ not dependent on $n$ and $x$, such that $(x-x^*)\tr \bar f_n(x) \ge \rho \left\| x_n - x\right\|$, where $\bar f_n(x) = \Upsilon(\overline H_n)^{-1} \nabla f(x)$.

\item $\{\xi_n, \xi_n^+,\xi_n^-, n=1,2,\ldots\}$ satisfy $\E\left[\left. \xi_n^+ + \xi_n^- - 2 \xi_n \right| \F_n\right] = 0$, for all $n$. 

\item  Same as (A4). %$\{d_n^i, i=1,\ldots,N, n=1,2,\ldots\}$ are i.i.d. and independent of $\F_n$.

\item Same as (A5).

\item For each $i=1,\ldots,N$ and any $\rho>0$, 
$P(\{ \bar f_{ni} (x_n) \ge 0 \text{ i.o}\} \cap \{ \bar f_{ni} (x_n) < 0 \text{ i.o}\} \mid \{ |x_{ni} - x^*_i| \ge \rho\quad \forall n\}) =0.$

\item The operator $\Upsilon$ satisfies $\delta_n^2 \Upsilon(H_n)^{-1} \rightarrow 0$ a.s. and  $E(\left\| \Upsilon(H_n)^{-1}\right\|^{2+\zeta}) \le \rho$ for some $\zeta, \rho>0$.

\item For any $\tau >0$ and nonempty $S \subseteq \{1,\ldots,N\}$, there exists a $\rho'(\tau,S)>\tau$ such that 
$$ \limsup_{n\rightarrow \infty} \left| \dfrac{\sum_{i \notin S} (x-x^*)_i \bar f_{ni}(x)}{\sum_{i \in S} (x-x^*)_i \bar f_{ni}(x)}               \right| < 1 \text{ a.s.}$$
for all $|(x-x^*)_i| < \tau$ when $i \notin S$ and $|(x-x^*)_i| \ge \rho'(\tau,S)$ when $i\in S$.
\item For some $\alpha_0, \alpha_1>0$ and for all $n$, $\E {\xi_n}^{2} \le \alpha_0$, $\E {\xi_n^{\pm}}^{2} \le \alpha_0$, $\E f(x_n)^{2} \le \alpha_1$ and $\E f(x_n\pm \delta_n d_n)^{2} \le \alpha_1$. 
\item  $\sum_n \frac{1}{(n+1)^{2}\delta_n^{2}} < \infty$.
\end{enumerate}
\begin{lemma}(\textbf{Bias in Hessian estimate})
\label{lemma:2rdsa-bias}
Under (C1)-(C10), with $\widehat H_n$ defined according to either \eqref{eq:2rdsa-estimate-unif} or \eqref{eq:2rdsa-estimate-ber}, we have a.s. that\footnote{Here $\widehat H_n(i,j)$ and $\nabla^2_{ij}f(\cdot)$ denote the $(i,j)$th entry in the Hessian estimate $\widehat H_n$ and the true Hessian $\nabla^2 f(\cdot)$, respectively.}, for $i,j = 1,\ldots,N$,
\begin{align}
\left|\E\left[
\left. \widehat H_n(i,j) \right| \F_n \right] - \nabla^2_{ij} f(x_n)\right| = O(\delta_n^2).
\end{align} 
\end{lemma}
\begin{proof}
Proof was given in \cite{prashanth2015rdsa}.
\end{proof}
\subsubsection{Finding feedback term}
Now consider the following 
\begin{align}
\widehat H_n & = M_n \left(\dfrac{y_n^+ + y_n^- - 2 y_n}{\delta_n^2}\right) \nonumber \\
& =  M_n \left[\left(\dfrac{f(x_n+\delta_n d_n) + f(x_n-\delta_n d_n) - 2 f(x_n)}{\delta_n^2}\right) \right.\nonumber \\&\hspace{10em}+\left. \left(\dfrac{\xi_n^+ + \xi_n^- - 2 \xi_n}{\delta_n^2}\right)\right] \nonumber \\
& = M_n \left(d_n\tr \nabla^2 f(x_n) d_n +  O(\delta_n^2) + \left(\dfrac{\xi_n^+ + \xi_n^- - 2 \xi_n}{\delta_n^2} \right)\right)\nonumber \\
& = M_n \left(d_n\tr \nabla^2 f(x_n) d_n +  O(\delta_n^2) + O(\delta_n^{-2})\right). \label{eq:h0}
\end{align}
From the above equation and using the fact that $\E\left[\left.\xi_n^+ + \xi_n^- - 2\xi_n \right| \F_n\right] = 0$, now we get
\begin{align}
\E[\widehat H_n \mid \F_n] = &  \E\left[\left. M_n \left(\sum\limits_{i=1}^{N-1} (d_n^i)^2 \nabla^2_{ii} f(x_n) \right.\right.\right.\nonumber\\
&\left.\left.\left.+ 2\sum\limits_{i=1}^N\sum\limits_{j=i+1}^N d_n^i d_n^j \nabla^2_{ij} f(x_n) + O(\delta_n^2)\right)\right| \F_n\right]. \label{eq:h1}
\end{align}

In analyzing the $lth$ diagonal term of above expression, the following term

\begin{align}
& \E\left[\left.\dfrac{45}{4\eta^4} \left((d_n^l)^2-\frac{\eta^2}{3}\right) \left(2\sum\limits_{i=1}^{N-1}\sum\limits_{j=i+1}^N d_n^i d_n^j \nabla^2_{ij} f(x_n)\right)\right| \F_{n}\right]\nonumber \\ &= 0.\label{eq:h2}
\end{align}
The matrix form for the above term is as follows 
\begin{align}\label{eq:digmat}
\Psi_{n}^{1}(\nabla^2 f(x_n)) = [M_n]_{D}\left(d_{n}\tr \, [\nabla^2 f(x_n)]_{N} \, d_{n}\right).
\end{align}
where for a matrix $[.]_{D}$ refers to keeping all its diagonal elements intact and making all its remaining elements zero. vice versa for $[.]_{N}$.\\

In analyzing the off-diagonal term($(k,l)$ where $k < l$) of the expression \eqref{eq:h1}, the following term
\begin{align}
& \dfrac{9}{2\eta^4} \E\left[\left.d_n^k d_n^l   \left(\sum\limits_{i=1}^N (d_n^i)^2 \nabla^2_{ii} f(x_n)\right)\right| \F_n \right] = 0\nonumber\\
\end{align}
The matrix notation for the above expression is as follows
\begin{align}\label{eq:ndmat}
\Psi_{n}^{2}(\nabla^2 f(x_n)) = [M_n]_{N}\left(d_{n}\tr \, [\nabla^2 f(x_n)]_{D} \, d_{n}\right).
\end{align}
Now consider the following
\begin{align}
\Psi_{n}(H) &= \Psi_{n}^{1}(H) + \Psi_{n}^{2}(H)\nonumber\\
&= [M_n]_{D}\left(d_{n}\tr \, [H]_{N} \, d_{n}\right) +  [M_n]_{N}\left(d_{n}\tr \, [H]_{D} \, d_{n}\right).
\end{align}
The feedback term presented in \eqref{eq:e2rdsa} is of the following form
\begin{align}
\widehat \Psi_n = \Psi_{n} (\overline H_{n-1}).
\end{align}
\subsubsection{Finding optimal wieghts}
\begin{align}
\label{eq:hess}
\overline H_n = \sum\limits_{i=0}^{n} w_k^{(n)}(\widehat H_i -\widehat \Psi_i).
\end{align}
where $w_i^{(n)}$ satisfy the following conditions $w_i^{(n)} \geq 0 ~\forall i$ and $\sum \limits_{i=0}^{n}w_i^{(n)} = 1$. The dominant term which contributes to the asymptotic variance is $O(\delta_n^{-2})$ term in the equation \eqref{eq:h0}. So we need to find the $w_i^{(n)},i=0,1,\ldots,n$ that minimizes the equation $\sum \limits_{i=0}^{n} (w_k^{(n)})^2 \delta_i^{-4}$ subjected to conditions mentioned earlier. The optimal $w_i^{(n)}$ after solving the above constrained optimization problem is of the form $w_i^{(n)} = \dfrac{\delta_i^{-4}}{\sum \limits_{j=0}^{n} \delta_j^{-4}} ~\forall i=0,1,\ldots,n$. The weights $w_i$ defined in \eqref{eq:2rdsa-H}
can be found by $w_i^{(n)}$. Which are of the form 
\begin{align}
\label{eq:wieghts}
w_i  = \dfrac{\delta_i^{-4}}{\sum\limits_{j=0}^{i} \delta_j^{-4} } .
\end{align}

















\section{CONCLUSIONS}


\addtolength{\textheight}{-12cm}   % This command serves to balance the column lengths
                                  % on the last page of the document manually. It shortens
                                  % the textheight of the last page by a suitable amount.
                                  % This command does not take effect until the next page
                                  % so it should come on the page before the last. Make
                                  % sure that you do not shorten the textheight too much.

%%%%%%%%%%%%%%%%%%%%%%%%%%%%%%%%%%%%%%%%%%%%%%%%%%%%%%%%%%%%%%%%%%%%%%%%%%%%%%%%



%%%%%%%%%%%%%%%%%%%%%%%%%%%%%%%%%%%%%%%%%%%%%%%%%%%%%%%%%%%%%%%%%%%%%%%%%%%%%%%%



%%%%%%%%%%%%%%%%%%%%%%%%%%%%%%%%%%%%%%%%%%%%%%%%%%%%%%%%%%%%%%%%%%%%%%%%%%%%%%%%
\section*{APPENDIX}



\section*{ACKNOWLEDGMENT}





%%%%%%%%%%%%%%%%%%%%%%%%%%%%%%%%%%%%%%%%%%%%%%%%%%%%%%%%%%%%%%%%%%%%%%%%%%%%%%%%





\bstctlcite{IEEEexample:BSTcontrol}
\bibliographystyle{IEEEtran}
\bibliography{reference}



\end{document}
