We make the same assumptions as those used in the analysis of \cite{prashanth2015rdsa}, with a few minor alterations. The assumptions are listed below:
\begin{enumerate}[label=(\textbf{C\arabic*})]
\item  The function
$f$ is four-times differentiable\footnote{Here $\nabla^4 f(x) = \dfrac{\partial^4 f (x)}{\partial x\tr \partial x\tr \partial x\tr \partial x\tr}$ denotes the fourth derivate of $f$ at $x$ and $\nabla^4_{i_1 i_2 i_3 i_4} f(x)$ denotes the $(i_1 i_2 i_3 i_4)$th entry of $\nabla^4 f(x)$, for $i_1, i_2, i_3,i_4=1,\ldots, N$.} with $\left|\nabla^4_{i_1 i_2 i_3 i_4} f(x) \right| < \infty$, for $i_1, i_2, i_3,i_4=1,\ldots, N$ and for all $x\in \R^N$. 

%\item  For some $\rho>0$  and almost all $x_n$, the function $f$ is four-times differentiable with a uniformly (in $n$) bounded fourth derivative for all $x$ such that $\left\| x_n - x\right\| \le \rho$. 

\item For each $n$ and all $x$, there exists a $\rho>0$ not dependent on $n$ and $x$, such that $(x-x^*)\tr \bar f_n(x) \ge \rho \left\| x_n - x\right\|$, where $\bar f_n(x) = \Upsilon(\overline H_n)^{-1} \nabla f(x)$.

\item $\{\xi_n, \xi_n^+,\xi_n^-, n=1,2,\ldots\}$ satisfy $\E\left[\left. \xi_n^+ + \xi_n^- - 2 \xi_n \right| \F_n\right] = 0$, for all $n$, where $\mathcal{F}_n = \sigma(x_m,m\le n)$ denotes the underlying sigma-field.. 

\item $\{d_n^i, i=1,\ldots,N, n=1,2,\ldots\}$ are i.i.d. and independent of $\F_n$.

\item  The step-sizes $a_n$ and perturbation constants $\delta_n$ are positive, for all $n$ and satisfy
$$\hspace{-2em} a_n, \delta_n \rightarrow 0\text{ as } n \rightarrow \infty, 
\sum_n a_n=\infty \text{ and } \sum_n \left(\frac{a_n}{\delta_n}\right)^2 <\infty.$$

\item For each $i=1,\ldots,N$ and any $\rho>0$, 
$P(\{ \bar f_{ni} (x_n) \ge 0 \text{ i.o}\} \cap \{ \bar f_{ni} (x_n) < 0 \text{ i.o}\} \mid \{ |x_{ni} - x^*_i| \ge \rho\quad \forall n\}) =0.$

\item The operator $\Upsilon$ satisfies $\delta_n^2 \Upsilon(H_n)^{-1} \rightarrow 0$ a.s. and  $E(\left\| \Upsilon(H_n)^{-1}\right\|^{2+\zeta}) \le \rho$ for some $\zeta, \rho>0$.

\item For any $\tau >0$ and nonempty $S \subseteq \{1,\ldots,N\}$, there exists a $\rho'(\tau,S)>\tau$ such that 
$$ \limsup_{n\rightarrow \infty} \left| \dfrac{\sum_{i \notin S} (x-x^*)_i \bar f_{ni}(x)}{\sum_{i \in S} (x-x^*)_i \bar f_{ni}(x)}               \right| < 1 \text{ a.s.}$$
for all $|(x-x^*)_i| < \tau$ when $i \notin S$ and $|(x-x^*)_i| \ge \rho'(\tau,S)$ when $i\in S$.
\item For some $\alpha_0, \alpha_1>0$ and for all $n$, $\E {\xi_n}^{2} \le \alpha_0$, $\E {\xi_n^{\pm}}^{2} \le \alpha_0$, $\E f(x_n)^{2} \le \alpha_1$,  $\E f(x_n\pm \delta_n d_n)^{2} \le \alpha_1$ and $\E \left(\left\| \Upsilon(\overline H_n \right\|^2 \mid \F_n\right) \le \alpha_1$. 
\item  $\delta_n = \frac{\delta_0}{(n+1)^{\varsigma}}$, where $\delta_0 > 0$ and $0 < \varsigma \le 1/8$.
\end{enumerate}
The reader is referred to Section II-B of \cite{prashanth2015rdsa} for a detailed discussion of the above assumptions. We remark here that (C1)-(C8) are identical to that in \cite{prashanth2015rdsa}, while (C9) and (C10) introduce minor additional requirements on $\left\| \Upsilon(\overline H_n \right\|^2$ and $\delta_n$, respectively and these are inspired by \cite{spall-jacobian}.

\begin{lemma}(\textbf{Bias in Hessian estimate})
\label{lemma:2rdsa-bias}
Under (C1)-(C10), with $\widehat H_n$ defined according to either \eqref{eq:2rdsa-estimate-unif} or \eqref{eq:2rdsa-estimate-ber}, we have a.s. that\footnote{Here $\widehat H_n(i,j)$ and $\nabla^2_{ij}f(\cdot)$ denote the $(i,j)$th entry in the Hessian estimate $\widehat H_n$ and the true Hessian $\nabla^2 f(\cdot)$, respectively.}, for $i,j = 1,\ldots,N$,
\begin{align}
\left|\E\left[
\left. \widehat H_n(i,j) \right| \F_n \right] - \nabla^2_{ij} f(x_n)\right| = O(\delta_n^2).
\end{align} 
\end{lemma}
\begin{proof}
See Lemma 4 in \cite{prashanth2015rdsa}.
\end{proof}

\begin{theorem}(\textbf{Strong Convergence of Hessian})
\label{thm:2rdsa-H}
Under (C1)-(C10), we have that 
$$\overline H_n \rightarrow \nabla^2 f(x^*) \text{ a.s. as } n\rightarrow \infty.$$ 
In the above, $\overline H_n$ is updated according to \eqref{eq:2rdsa-H} and $\widehat H_n$ defined according to either \eqref{eq:2rdsa-estimate-ber} or \eqref{eq:2rdsa-estimate-unif}. 
\end{theorem}
\begin{proof}

For proving the main claim regarding $\overline H_n$, we closely follow the approach used to prove a corresponding result for 2SPSA (see Theorem 1 in \cite{spall-jacobian}). 
The first step is to prove the following:
\begin{align}
\sum_{k=0}^n \dfrac{\delta_k^4 \left(\widehat H_k - \widehat \Psi_k - \E(\widehat H_k \mid \F_k)\right)}{\sum_{i=0}^n \delta_i^4} \rightarrow 0.
\label{eq:step1}
\end{align}

By a completely parallel argument to that used in the proof of Theorem 1 in \cite{spall-jacobian}, we obtain: For any $i,j = 1,\ldots,N$,
\begin{align*}
\E \left[\left((\widehat {H_k})_{i,j} - (\widehat {\Psi_k})_{i,j} - \E((\widehat {H_k})_{i,j} \mid \F_k)\right)^2\right] = O(\delta_k^{-4}).
\end{align*}
Now \eqref{eq:step1} follows by an application of Kronecker's Lemma along with the martingale convergence theorem (see Theorem 6.2.1 of \cite{lahaprobability}).

From Lemma \ref{lemma:2rdsa-bias}, we have 
$$ \E[ \widehat H_k \mid \F_k] = \nabla^2 f(x_n) + O(\delta_n^2) \text{ a.s.}$$
Since the Hessian is continuous near $x_n$ and $x_n$ converges almost surely to $x^*$, we have
\footnote{Since assumptions (C1)-(C10) here are similar to that in \cite{prashanth2015rdsa}, Theorem 5 of \cite{prashanth2015rdsa} holds here as well. In particular, this implies almost sure convergence of $x_n$ to $x^*$.}
\begin{align*}
\sum_{k=0}^n \dfrac{\delta_k^4 \left(\E(\widehat H_k \mid \F_k)\right)}{\sum_{i=0}^n \delta_i^4} 
=&\sum_{k=0}^n \dfrac{\delta_k^4 \left(\nabla^2 f(x_n) + O(\delta_n^2)\right)}{\sum_{i=0}^n \delta_i^4}\\
=&\sum_{k=0}^n \dfrac{\delta_k^4 \left(\nabla^2 f(x^*) + o(1)\right)}{\sum_{i=0}^n \delta_i^4}\\
&\rightarrow \nabla^2 f(x^*) \text{ a.s. as } n \rightarrow \infty.
\end{align*}
The last step above follows from Toeplitz Lemma (see p. 89 of \cite{lahaprobability}) after observing that $\sum_{i=0}^n \delta_i^4 \rightarrow \infty$ due to (C10). 
The main claim now follows since 
$$ \overline H_n = \sum_{k=0}^n \dfrac{\delta_k^4 \left(\widehat H_k - \Psi_k \right)}{\sum_{i=0}^n \delta_i^4}.$$
\end{proof}


%%%%%%%%%%%%%%%%%%%%%%%%%%%%%%%%%%%%%%%%%%%%%%%%%%%%5
%%%%%%%%%%%%%%%%%%%%%%% Quadratic case
%%%%%%%%%%%%%%%%%%%%%%%%%%%%%%%%%%%%%%%%%%%%%%%%%%%%


\begin{align}
 \widehat H_n =    \Phi_n(\nabla^2 f(x_n)) &+\Psi_{n}(\nabla^2 f(x_n)) +  O(\delta_n^2)\nonumber\\
&+\left(\dfrac{\xi_n^+ + \xi_n^- - 2 \xi_n}{\delta_n^2} \right) \label{eq:hnhat-ext}
\end{align}
Where, for any matrix $H$, 
\begin{align}
&\Phi_{n}(H) = [M_n]_{N}\left(d_{n}\tr \, [H]_{D} \, d_{n}\right) +  [M_n]_{D}\left(d_{n}\tr \, [H]_{N} \, d_{n}\right).
\end{align}
And $\Psi_n(H)$ is defined in the equation \eqref{eq:psi}.
\begin{theorem}(\textbf{})
\label{thm:quad-bound}
Suppose $f$ is quadratic function and only noise-free measurements  of $f$ are used to form $\widehat H_n$ in \eqref{eq:2rdsa-estimate-ber}. Suppose $0 < w_0\leq1$ and $w_n = w/n^r$, $n=1,2,\ldots,k$, where $1/2 < r < 1$ and $0 < w \leq 1$. Suppose the  conditions $\ldots\ldots\ldots\ldots\ldots\ldots\ldots$ hold and $d_n$ are identically distributed for each $n$. Further, Suppose $\nabla^2 f(x^*) > 0$ and $\Upsilon$ in \eqref{eq:e2rdsa} is such that $\E(\parallel \Upsilon(\overline H_n) - \overline H_n\parallel^2) = o(e^{-2wn^{1-r}/(1-r)})$ and $\parallel \Upsilon(H) - H \parallel^2 / (1+\parallel H \parallel^2)$ is uniformly bounded with respect to the set of symmetric $H \in \mathbb{R}^{p \times p}$. Then $trace[\E (\Lambda_n \tr \Lambda_n)] = O(e^{-2wn^{1-r} / {1-r}})$. Where $\Lambda_k = \overline H_k - H^*$
\end{theorem}
\begin{proof}
The proof consists of three parts. Those are as folllows
\begin{enumerate}[]
  \item proof of MSE convergence of $\overline H_k$.
  \item derivation of convenient representation of $trace[\E (\Lambda_n \tr \Lambda_n)]$.
  \item derivation of main big-$O$ result.
\end{enumerate}
\emph{$Part(i)$ : MSE convergence of $\overline H_n$ :} This part follows from theorem 3 of \cite{spall-jacobian} .\\
\emph{$Part(ii)$ :Representation of $trace[\E (\Lambda_n \tr \Lambda_n)]$ :} Let $\Lambda_n' = \Upsilon(\overline H_k) - H^*$. From \eqref{eq:2rdsa-H} we can write
\begin{align}\label{eq:2rdsa-quadthm}
\overline H_n = \overline H_{n-1} - w_n (\overline H_{n-1} - \hat H_n + \hat \Psi_n).
\end{align}
 From the equations \eqref{eq:2rdsa-quadthm} and using the fact for the quadratic case $\hat H_n = \Phi_n(H^*) + \Psi_n(H^*)$ ,  we can write 
 \begin{align}\label{eq:lambdak}
 \Lambda_n &= \Lambda_{n-1} - w_n ( H_{n-1} - \hat H_n + \hat \Psi_n) \nonumber\\&= (1-w_n) \Lambda_{n-1} - w_n (H^* + \hat \Psi_n - \hat H_n ) \nonumber\\&= (1-w_n) \Lambda_{n-1} - w_n (H^*+\hat \Psi_n -\Phi_n(H^*) - \Psi_n(H^*)\nonumber\\&= (1-w_n) \Lambda_{n-1} - w_n \Psi_n(\Upsilon(\overline H_{n-1})) \nonumber\\& \hspace{2.6cm}+ w_n (\Phi_k(H^*) - H^*)\nonumber\\&= (1-w_n) \Lambda_{n-1} - w_n\Psi_n(\Lambda_{n-1}')+ w_n (\Phi_k(H^*) - H^*)
 \end{align}
 Where last equality followed from the form of $\Psi(.)$ in the equation \eqref{eq:psi}. Then
 \begin{align}\label{lambda-exp}
 \Lambda_n  = & \left[ \prod_{k=1}^n (1-w_k) \right]\Lambda_0 \nonumber\\ &- \sum_{k=1}^n \left[\prod_{j=k+1}^n (1-w_j)\right] w_k \Psi_k(\Lambda_{k-1}') \nonumber\\ &+ \sum_{k=1}^n \left[\prod_{j=k+1}^n (1-w_j)\right] w_k (\Phi_k(H^*) - H^*) \hspace{0.2cm} a.s.
 \end{align}
 (note : $\prod_{j=n+1}^n (1-w_j) = 1$ for all n).
 Let us  characterise $trace[\E(\Lambda_n\tr \Lambda_n)]$ using \eqref{lambda-exp}. From independence of $d_k$ along k, \eqref{lambda-exp} represent martingale difference sequence, leading to 
 \begin{align}\label{eq:exp-lambda2}
 & \E (\Lambda_n \tr \Lambda_n) =  \left[ \prod_{k=1}^n (1-w_k) \right]^2 \E (\Lambda_0 \tr \Lambda_0)  \nonumber\\ &+ \sum_{k=1}^n \left[\prod_{j=k+1}^n (1-w_j)\right]^2 w_k^2   \E (\Psi_k(\Lambda_{k-1}') \tr \Psi_k(\Lambda_{k-1}'))  \nonumber\\ &+ \sum_{k=1}^n \left[\prod_{j=k+1}^n (1-w_j)\right]^2 w_k^2  \nonumber\\& \hspace{2cm} \times \E ((\Phi_k(H^*) - H^*) \tr  (\Phi_k(H^*) - H^*)) \nonumber\\&+\sum_{k=1}^n \left[\prod_{j=k+1}^n (1-w_j)\right]^2 w_k^2 \E\left(\Psi_k \tr (\Phi_k(H^*) - H^*))\right).
 \end{align}
 From the independence of $d_k$'s and $\Lambda_{k-1}'$ and using Cauchy-Schwartz inequality we get 
 \begin{align}\label{eq:trace}
 & trace \left[\E (\Lambda_n \tr \Lambda_n)\right] \leq \left[ \prod_{k=1}^n (1-w_k) \right]^2 trace \left[\E (\Lambda_0 \tr \Lambda_0)\right]  \nonumber\\ &+ \sum_{k=1}^n \left[\prod_{j=k+1}^n (1-w_j)\right]^2 w_k^2  \,\,\tau\left(\E (\Lambda_{k-1}' \otimes \Lambda_{k-1}')\right)  \nonumber\\ &+ \sum_{k=1}^n \left[\prod_{j=k+1}^n (1-w_j)\right]^2 w_k^2  \nonumber\\&  \hspace{0.5cm}\times trace\left[\E ((\Phi_k(H^*) - H^*) \tr  (\Phi_k(H^*) - H^*))\right] + \nonumber\\ &+ \sum_{k=1}^n \left[\prod_{j=k+1}^n (1-w_j)\right]^2 w_k^2   \tau\left(\E (\Lambda_{k-1}' \otimes \Lambda_{k-1}')\right)^{1/2} \nonumber\\&  \times trace\left[\E ((\Phi_k(H^*) - H^*) \tr  (\Phi_k(H^*) - H^*))\right]^{1/2}.
 \end{align}
Note that $1-w_k = e^{-w_k}(1-O(w_k^2))$, where the $O(w_k^2)$ term is strictly positive for $0 < w_k <1$ by the convexity of $e^{-w_k}$. Let $w(i,j) = \sum_{k=i}^j w_k$ and $a_{kn} = \left[\prod_{i=k+1}^n (1- O(w_i^2))\right]^2$ and $a_{nn} = 1$. Then
\begin{align}\label{eq:wij}
 & trace \left[\E (\Lambda_n \tr \Lambda_n)\right] \leq e^{-2 w(1,n)} a_{0n} trace \left[\E (\Lambda_0 \tr \Lambda_0)\right]  \nonumber\\ &+ e^{-2 w(1,n)} \sum_{k=1}^n e^{2 w(1,k)} a_{kn} w_k^2  \,\,\tau\left(\E (\Lambda_{k-1}' \otimes \Lambda_{k-1}')\right)  \nonumber\\ &+ e^{-2 w(1,n)} \sum_{k=1}^n e^{2 w(1,k)} a_{kn} w_k^2  \nonumber\\&  \hspace{0.5cm}\times trace\left[\E ((\Phi_k(H^*) - H^*) \tr  (\Phi_k(H^*) - H^*))\right] \nonumber\\ &+ e^{-2 w(1,n)} \sum_{k=1}^n e^{2 w(1,k)} a_{kn} w_k^2  \tau\left(\E (\Lambda_{k-1}' \otimes \Lambda_{k-1}')\right)^{1/2} \nonumber\\&  \times trace\left[\E ((\Phi_k(H^*) - H^*) \tr  (\Phi_k(H^*) - H^*))\right]^{1/2}.
 \end{align} 
From the hypothesis $0 < w_k < 1$ for all $k \geq 2$ and $r > 0.5$, the $a_{kn}$ are uniformly bounded in magnitude. Further by the facts $\sum_{k=i}^j w_k \to \infty$ as $j-i \to \infty$.
\begin{align}\label{eq:intwts}
w(i,j) = \int_i^j \frac{w}{x^r} dx + O(1) &= \left(\frac{w}{1-r}\right)(j^{1-r}-i^{1-r})\nonumber\\ &\hspace{2cm}+O(1)
\end{align}
From the \eqref{eq:wij} and \eqref{eq:intwts}
\begin{align}\label{eq:thmwts}
 & trace \left[\E (\Lambda_n \tr \Lambda_n)\right] =  e^{-2 w(1,n)} a_{0n} trace \left[\E (\Lambda_0 \tr \Lambda_0)\right]  \nonumber\\ &+ \bar a_n e^{-2 w n^{1-r}/(1-r)} \sum_{k=1}^n e^{2 w k^{1-r}/(1-r)}  \times \frac{w^2}{k^{2 r}} \nonumber\\& \hspace{3.5cm} \times \tau\left(\E (\Lambda_{k-1}' \otimes \Lambda_{k-1}')\right)  \nonumber\\ &+ \bar a_n e^{-2 w n^{1-r}/(1-r)} \sum_{k=1}^n e^{2 w k^{1-r}/(1-r)}  \times \frac{w^2}{k^{2 r}}  \nonumber\\&  \hspace{0.5cm}\times trace\left[\E ((\Phi_k(H^*) - H^*) \tr  (\Phi_k(H^*) - H^*))\right] \nonumber\\ &+ \bar a_n e^{-2 w n^{1-r}/(1-r)} \sum_{k=1}^n e^{2 w k^{1-r}/(1-r)}  \times \frac{w^2}{k^{2 r}}  \nonumber\\&\hspace{3.5cm}\times \tau\left(\E (\Lambda_{k-1}' \otimes \Lambda_{k-1}')\right)^{1/2}\nonumber\\&\times trace\left[\E ((\Phi_k(H^*) - H^*) \tr  (\Phi_k(H^*) - H^*))\right]^{1/2}.
\end{align} $\bar a_n$ is uniformly bounded in magnitude by the corresponding uniform boundedness of the $a_{kn}$.\\
\emph{$Part(iii)$ The big-O result on rate of convergence:}
 We have two extra terms term three and four for \eqref{eq:intwts} when compared to \cite{spall-jacobian}. Let us consider third term, we know that $trace\left[\E ((\Phi_k(H^*) - H^*) \tr  (\Phi_k(H^*) - H^*))\right]$ bounded uniformly for for every $k$. Then  $\bar b_n e^{-2 w n^{1-r}/(1-r)} \sum_{k=1}^n e^{2 w k^{1-r}/(1-r)}  \times \frac{w^2}{k^{2 r}} \to 0$ as $n \to \infty$ by Kronecker's lemma and $\sum_{k=1}^n \frac{1}{k^{2 r}} < \infty$ because $1/2 < r < 1$.  Again by using uniform boundedness of $trace\left[\E ((\Phi_k(H^*) - H^*) \tr  (\Phi_k(H^*) - H^*))\right]$ we can rewrite  the fourth term in \eqref{eq:intwts} as follows 
 \begin{align}\label{eq:fourthterm}
 \bar b_n e^{-2 w n^{1-r}/(1-r)} \sum_{k=1}^n e^{2 w k^{1-r}/(1-r)}  \frac{w^2}{k^{2 r}} \nonumber \\ \hspace{3cm}\times \tau\left(\E (\Lambda_{k-1}' \otimes \Lambda_{k-1}')\right)^{1/2}
 \end{align}
 Hence the contradiction proof given in theorem 3,part (iii) of \cite{spall-jacobian} will follow from now on by using the fact that term in \eqref{eq:fourthterm} is $O(second term)$ of \eqref{eq:thmwts}. 
 
 
\end{proof}